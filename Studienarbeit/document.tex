\documentclass[4paper,10pt]{article} 
\usepackage[a4paper,
  left=2.6cm, right=2.6cm,
  top=2.8cm, bottom=2.8cm]{geometry}

\usepackage{selinput} 
\SelectInputMappings{ 
adieresis={ä}, 
germandbls={ß}} 
\usepackage[ngerman]{babel} 
\usepackage{graphicx}
\usepackage{fancyhdr}
\pagestyle{fancy}
\fancyhf{}



%Kopf- und Fußzeile
\pagestyle{fancy} 
\fancyhf{} 

\fancyhead[L]{\leftmark} %Kopfzeile links
\fancyhead[C]{} %zentrierte Kopfzeile
\fancyhead[R]{\includegraphics[width=0.1\textwidth]{./abbildungen/titelseite/dhbw_logo.png}}%Kopfzeile rechts
\fancyfoot[R]{\thepage} %Seitennummer
\fancyfoot[L]{} %Seitennummer
\fancyfoot[C]{Jörg Woditschka | Jan Brodhaecker} %Seitennummer


\newcommand{\superscript}[1]{\ensuremath{^{\textrm{#1}}}}
\newcommand{\subscript}[1]{\ensuremath{_{\textrm{#1}}}}
\newcommand{\hk}[1]{\grq}
\renewcommand{\footrulewidth}{0.4pt} 


%------------------------------------------------
%Beginn des Dokumentes
%------------------------------------------------
\begin{document}


\begin{titlepage}
\begin{center}
\includegraphics[width=0.5\textwidth]{./abbildungen/titelseite/dhbw_logo.png}\\[1cm]
\textsc{\Large Studienarbeit}\\[0.5cm]
\newcommand{\HRule}{\rule{\linewidth}{0.3mm}}
\HRule \\[0.3cm]{ \huge \bfseries Entwicklung einer OBD2-Android Applikation}\\[0.3cm]
\HRule \\[1.5cm]
\begin{minipage}{0.4\textwidth}
\begin{flushleft} \large
\center{
Jörg Woditschka \\
TINF11AI-BC \\
\# 1234456 \\
jwoditschka@gmail.com
}
\end{flushleft}
\end{minipage}
\hfill
\begin{minipage}{0.4\textwidth}
\begin{flushright} \large
\center{
Jan Brodhaecker \\
TINF11AI-BC \\
\# 3498577 \\
jan.brodhaecker@gmx.net
}
\end{flushright}
\end{minipage}
\vfill
{\large 30.Irgendwann.2014}
\end{center}
\end{titlepage}


\renewcommand{\thesection}{\Roman{section}} 
\section{Zusammenfassung}
Irgendein Inhalt \ldots 
\newpage


\section{Eidesstaatliche Erklärung}
Irgendein Inhalt \ldots
\newpage


\tableofcontents
\newpage

\section{Abbildungsverzeichnis}
\listoffigures
\newpage

\section{Tabellenverzeichnis}
\listoftables
\newpage

\section{Abkürzungsverzeichnis}
\newpage





\setcounter{section}{0} 
\renewcommand{\thesection}{\arabic{section}}


\section{Einleitung}
\newpage

\section{Abgrenzung des Projektes}
\newpage

\section{Grundlagen}
\subsection{OBD2-Schnittstelle}
\subsection{Phonegap}
Phonegap ist ein Framework der Firma Adobe, das zur Erstellung von
multi-plattform Applikationen für mobile Endgeräte genutzt werden kann. Dabei
wird im Grunde mit Hilfe von JavasScript, HTML und CSS eine Webapplikation
definiert, die auf die verschiedensten Plattformen portiert werden kann. Dabei erstellt Phonegap
allerdings keinen nativen Code für das jeweilige Gerät, sondern verknüft die
geschriebene Webapplikation nur mit einer vordefinierten nativen Funktion, die
die Webapplikation ledeglich startet. Zum heutigen Standpunkt\footnote{Phonegap
3.10} unterstützt Phonegap die folgenden mobilen Betriebssysteme : 
\begin{itemize}
		\item iOS (Apple)
		\item Android (Google)
		\item webOS (HP)
		\item Symbian OS (Nokia)
		\item Blackberry (Blackberry)
		\item Windows Phone (Microsoft).
	\end{itemize}
Da eine Phonegap-Applikation im Grunde nur mit Hilfe von JavaScript,
Hyper-Transfer-Meta-Language (HTML) 5 und Cascading-Style-Sheet(CSS) 3
entwickelt werden kann, fallen natürlich einige gerätespezifische Funktionen weg, so kann man beispielsweise mit JavaScript oder HTML5 nicht auf
die Telefon-Funktion eines Smartphones zugreifen, oder auf den SMS-Speicher. Mit
HTML5 kamen einige bahnbrechende Erneurungen, so kann mittlerweile per HTML auf
die Kamera, auf den Kompass oder auf lokale Dateisystem eines Smartphones oder
eines Tablets zugreifen. Wie bereits erwähnt werden auch mit HTML5 noch nicht
alle Funktionen, die ein Smartphone oder ein Tablet bietet abgedeckt - Phonegap
bietet dafür die Möglichkeit an, sogenannte Plugins zu verwenden oder gar zu
selbst zu entwickeln. \\
Bei einem Plugin kann man über speziell definierte Schnittstellen per JavaScript
auf nativen Code zugreifen. Dafür muss ein Plugin speziell in der
Phonegap-Applikation registriert werden. Der native Code muss dabei in einer
speziellen Form vorliegen und gewisse Funktionsnamen bieten, die dann später per
Javascript aufgerufen werden können. So kann man auch komplexe Applikationen mit
Hilfe von Phonegap relativ schnell entwickeln, dabei muss natürlich der native
Code für jede Plattform einzeln angepasst werden. Dagegen bleibt das
Benutzer-Interface allerdings für alle Plattformen gleich. \\
Zur Zeit gibt es auf dem Markt viele Frameworks, die die gleichen oder ähnlichen
Funktionen wie Phonegap bieten. Beispielsweise SenchaTouch (Sencha), RhoMobile
(Motorola) oder MoSync (MoSync). Da SenchaTouch mit kompletten Funktionsumfang
nicht komplett kostenfrei, fiel dieses Framework ziemlich schnell aus der Wahl.
Bei RhoMobile und MoSync führten die deutlich kleinere hilfegebende Community
und die Instabilität zum Ausscheiden. Deshalb fiel die Wahl schlussendlich auf
Phongeap. Phonegap unterliegt der Apache-Lizenz, was sie frei und veränderbar
macht, solange der veränderte Quellcode ebenfalls der Apache-Lizenz unterliegt
und für die Gesamtheit frei zugänglich bleibt.



\subsection{Vergleich zu anderen Applikationen}
Um bereits existierende Applikationen möglichst objektiv miteinander vergleichen
zu können, wurde ein kleiner Fragenkatalog entwickelt, der schlussendlich auch
zur Orientierung diente. Dabei wurden der Funktionsumfang, das
Benutzer-Interface, der Preis und die Verfügbarkeit für die verschiedenen
mobilen Betriebssysteme.
\\ 


\begin{tabular}{ l | l | p{5cm} | l | l}

Name & Funktionsumfang & Benutzer-Interface & Preis & OS \\
\hline
\\
Torque&  blablabla& blallb & 3,50€ & Android \\


\end{tabular}



\subsection{rechtliche Situation}
\newpage


\section{Projektergebnisse}
\newpage

\section{Fazit}
\newpage





\setcounter{section}{5} 
\renewcommand{\thesection}{\Roman{section}}

\section{\hspace{0.2cm}Literatur}
\newpage

\section{\hspace{0.1cm}Anhang}
\newpage


















\end{document}
